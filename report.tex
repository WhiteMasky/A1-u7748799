\documentclass[12pt]{article}

\usepackage{times} % Times New Roman font
\usepackage{graphicx} % For images
\usepackage{enumitem} % For numbered items
\usepackage{lipsum} % For dummy text
\usepackage{hyperref} % For references
\usepackage{fancyhdr} % For custom headers
\usepackage{pythonhighlight}

\pagestyle{fancy}
\fancyhf{}
\rhead{Assignment 1\\COMP6528\\Computer Vision\\24S1\\u7748799\\Yichi Zhang}
\chead{}
\lhead{}
\renewcommand{\headrulewidth}{0pt}

\begin{document}

\section{Task 1: Basic Image I/O}

\subsection{Question 1}
\underline{All the runnable code will be provided in the code file. Hereinafter same.}\\
\\
\quad According to section 1.1 in the code file, we can complete the save operation according to the following code.
\begin{python}
# Load the provided images 
# (I change their name to XXXX_original.jpg to avoid confusion)
image1 = cv2.imread('image1_original.jpg')
image2 = cv2.imread('image2_original.jpg')
image3 = cv2.imread('image3_original.jpg')

# Resize the images to 1024 x 720
image1 = cv2.resize(image1, (1024, 720))
image2 = cv2.resize(image2, (1024, 720))
image3 = cv2.resize(image3, (1024, 720))

# Save the resized images as JPG files
cv2.imwrite('image1.jpg', image1)
cv2.imwrite('image2.jpg', image2)
cv2.imwrite('image3.jpg', image3)
\end{python}
% Leave blank if not responding to a particular question or task

\subsection{Question 2}

% Documentation, observations, results, analysis, etc.
\lipsum[1] % Dummy text

% Insert an image
\begin{figure}[h]
\centering
\includegraphics[width=0.5\textwidth]{example-image}
\caption{Descriptive caption}
\label{fig:example}
\end{figure}

% Refer to the image
As shown in Figure \ref{fig:example}, the results indicate that...

% References
\begin{thebibliography}{9}
\bibitem{example}
  Author,
  \emph{Title}.
  Publisher, Year.
\end{thebibliography}

\end{document}
